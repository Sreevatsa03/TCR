\chapter{Conclusion and Future Work}\label{sec:Conclusion}
\section{Conclusions}
The $V(D)J$ recombination process in TCRs offers a diverse set of receptors, which is necessary to facilitate T-cell responses to foreign invaders. In addition, scrutiny of the TCR repertoire enable immunologists to understand the functionality of healthy immune system, determine the nature of successful and unsuccessful immune responses, and understand the immune mechanism in presence of different diseases. The response of immune system to specific antigen often leaves evidence in the form of repertoire sequence signatures that are common across individuals and these signature patterns can be associated with the corresponding antigen. Identification of these signatures help immunologists to understand the correlation between the immune receptors and different disease, which provides researchers the ability to identify immune receptor clones that can be converted into precision vaccines. However, analysis of TCR pool require modeling of the diverse set of TCR, which is computationally challenging as the total number of TCRs to be generated and processed can exceed $10^{18}$ sequences. This massive scale of data processing poses as the barrier for immunologists to successfully understand the functionality of human immune system. Therefore, reducing the timescale of modeling the TCR repertoire is crucial for the immunologists. 

In this dissertation, we introduced a bit-wise implementation of the $V(D)J$ recombination algorithm, which reduces the constant memory and global memory footprint by factors of $3.4\times$ and $4\times$ respectively. On a single GPU, the bit-wise implementation reduces the total execution time by a factor of $2.1\times$ compared to the baseline implementation. We presented the multi-GPU version of the bit-wise recombination and conducted availability analysis. We showed that beyond n-nucleotide length of eight, since we fully occupy the thread blocks on a single GPU, we observed reduction in execution time with the increase in number of GPUs.  However this reduction shows a saturating trend.  We finally analyzed the root causes of observing a saturation trend in execution time reduction as we increase the GPU resources. As we transition from mouse data set to human data set, we expect the time scale of the experiments to increase by three orders of magnitude. In this scale, ability to reduce the simulation time from 40.5 hours to 18.9 hours on a single GPU and to 4.3 hours on a 8-GPU system for mouse data set is a significant gain that will allow us to  count the number of unique pathways a TCR sequence can be generated, and conduct statistical analysis to correlate those frequently generated TCR sequences to certain diseases much faster that the baseline version. 


We also mapped the $V(D)J$ recombination algorithm onto FPGA and take advantage of the fine grained parallelism offered by the target FPGA whose architecture naturally matches the program architecture of the recombination process. We first map the recombination algorithm using the N-level parallelization approach, which is used for GPU-based implementation. We show that this implementation suffers a large critical path delay and as a result low clock frequency. In order to address the draw back of N-level architecture, we propose the VJ-level parallelization approach. Simulation results showed that the total execution time reduces by a factor of $2.34\times$ in comparison with the N-level architecture for the FPGA-based implementation.
\section{Future Work}
Future work for the GPU-based implementation includes extending a bit-wise implementation of the $V(D)J$ recombination process to explore a data containing 50 million sequences from 100 humans, as well as additional data sets from mice and other species. One of the potential future work involves proposing scalable implementation of recombination process for the multi-GPU environment. The scalable GPU-based implementation will enable immunologists to analyze the human data set and provide them with a more solid understanding of the mechanisms that control the recombination process in the human immune system.

For the FPGA-based implementation, one potential future work would be utilizing dictionary based algorithm to speed up the comparison process, which can result in accelerating the entire process. Another work would be employing \emph{hash function} to eliminate the comparison step. This will significantly help us to increase the degree of parallelism due to elimination of comparators and significantly accelerate the process.



